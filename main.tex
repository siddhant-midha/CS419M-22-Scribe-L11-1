%
% This is the LaTeX template file for lecture notes for CS294-8,
% Computational Biology for Computer Scientists.  When preparing 
% LaTeX notes for this class, please use this template.
%
% To familiarize yourself with this template, the body contains
% some examples of its use.  Look them over.  Then you can
% run LaTeX on this file.  After you have LaTeXed this file then
% you can look over the result either by printing it out with
% dvips or using xdvi.
%
% This template is based on the template for Prof. Sinclair's CS 270.

\documentclass[11pt, twosides]{article}
\usepackage[utf8]{inputenc}
\usepackage{graphicx}
\usepackage{graphics}
\usepackage{amsmath}
\usepackage{amsfonts}
\usepackage{amssymb}
\usepackage{amsthm}
\usepackage{xcolor}
\setlength{\oddsidemargin}{0.25 in}
\setlength{\evensidemargin}{-0.25 in}
\setlength{\topmargin}{-0.6 in}
\setlength{\textwidth}{6.5 in}
\setlength{\textheight}{8.5 in}
\setlength{\headsep}{0.75 in}
\setlength{\parindent}{0 in}
\setlength{\parskip}{0.1 in}

%
% The following commands set up the lecnum (lecture number)
% counter and make various numbering schemes work relative
% to the lecture number.
%
\newcounter{lecnum}
\renewcommand{\thepage}{\thelecnum-\arabic{page}}
\renewcommand{\thesection}{\thelecnum.\arabic{section}}
\renewcommand{\theequation}{\thelecnum.\arabic{equation}}
\renewcommand{\thefigure}{\thelecnum.\arabic{figure}}
\renewcommand{\thetable}{\thelecnum.\arabic{table}}

%
% The following macro is used to generate the header.
%
\newcommand{\lecture}[4]{
%   \pagestyle{myheadings}
   \thispagestyle{plain}
   \newpage
   \setcounter{lecnum}{#1}
   \setcounter{page}{1}
   \noindent
   \begin{center}
   \framebox{
      \vbox{\vspace{2mm}
    \hbox to 6.28in { {\bf CS 419M Introduction to Machine Learning
                        \hfill Spring 2021-22} }
       \vspace{4mm}
       \hbox to 6.28in { {\Large \hfill Lecture #1: #2  \hfill} }
       \vspace{2mm}
       \hbox to 6.28in { {\it Lecturer: #3 \hfill Scribe: #4} }
      \vspace{2mm}}
   }
   \end{center}
   \markboth{Lecture #1: #2}{Lecture #1: #2}
}

%
% Convention for citations is authors' initials followed by the year.
% For example, to cite a paper by Leighton and Maggs you would type
% \cite{LM89}, and to cite a paper by Strassen you would type \cite{S69}.
% (To avoid bibliography problems, for now we redefine the \cite command.)
% Also commands that create a suitable format for the reference list.
% \renewcommand{\cite}[1]{[#1]}
% \def\beginrefs{\begin{list}%
%         {[\arabic{equation}]}{\usecounter{equation}
%          \setlength{\leftmargin}{2.0truecm}\setlength{\labelsep}{0.4truecm}%
%          \setlength{\labelwidth}{1.6truecm}}}
% \def\endrefs{\end{list}}
% \def\bibentry#1{\item[\hbox{[#1]}]}

%Use this command for a figure; it puts a figure in wherever you want it.
%usage: \fig{NUMBER}{SPACE-IN-INCHES}{CAPTION}
% \newcommand{\fig}[3]{
% 			\vspace{#2}
% 			\begin{center}
% 			Figure \thelecnum.#1:~#3
% 			\end{center}
% 	}
% Use these for theorems, lemmas, proofs, etc.
\newtheorem{theorem}{Theorem}[lecnum]
\newtheorem{lemma}[theorem]{Lemma}
\newtheorem{proposition}[theorem]{Proposition}
\newtheorem{claim}[theorem]{Claim}
\newtheorem{corollary}[theorem]{Corollary}
\newtheorem{definition}[theorem]{Definition}
% \newenvironment{proof}{{\bf Proof:}}{\hfill\rule{2mm}{2mm}}

% **** IF YOU WANT TO DEFINE ADDITIONAL MACROS FOR YOURSELF, PUT THEM HERE:

\begin{document}
%FILL IN THE RIGHT INFO.
%\lecture{**LECTURE-NUMBER**}{**DATE**}{**LECTURER**}{**SCRIBE**}
\lecture{11}{}{Abir De}{Group 1}
%\lecture{x}{Title}{Abir De}{Group y}

\section{Stability of Ranking Loss}
We have the Ranking loss for all pairs of $x_{bad}$ and $x_{good}$ given by -
$${f(w) = \sum_{\substack{y_{bad}=-1 \\ y_{good}=+1 \\ x_{bad,good}\in S}}( 1 + w^{T}x_{bad} -  w^{T}x_{good})}$$

What should be the regulariser $\lambda$ that should be added to the loss to ensure that it is stable?

Suggestion - If we consider $x_{bad} - x_{good}$ as a new proxy variable , then the regulariser will be the ordinality of this, that is, the number of pairs of $x_{bad}$ and $x_{good}$. So the loss function becomes 
$${f(w) = \sum_{\substack{y_{bad}=-1 \\ y_{good}=+1 \\ x_{bad,good}\in S}}( 1 + w^{T}x_{bad} -  w^{T}x_{good}) + \lambda |S_{good}||S_{bad}|||w||^2}$$

However proving the stability for this is not the same as proving the stability for a single variable as before because here, when we change the set S by replacing an old point with a new point ${x,y}$ to get $S^{'}$, all pairs in the summation of which it is a part will change. Thus, this is different from a point-wise loss. 

Optimal w can be easily found for this case by differentiating the loss term wrt w to get

$${w = \sum\frac{(x_{good} - x_{bad})}{2\lambda|S_{good}||S_{bad}|}}$$

What would it be if the loss was calculated using the hinge function and what is the condition of stability?
$${f(w) = \sum_{\substack{y_{bad}=-1 \\ y_{good}=+1 \\ x_{bad,good}\in S}}( 1 + w^{T}x_{bad} -  w^{T}x_{good})_{+} + \lambda |S_{good}||S_{bad}|||w||^2}$$

Following the same line of proof as earlier, we will get some bound in terms of ${\frac{1}{|S_{good}|} or \frac{1}{|S_{bad}|}}$

But can we find some regulariser that is independent of the ratio of $x_{good}$ to $x_{bad}$?

Solution - If we use the following regulariser, though we would be over-regularising w but we can be ensured stability.

$${\frac {|S|(|S|-1)}{2}}$$


\section{Group Details and Individual Contribution}
% Fill this part
\end{document}





